\chapter[Conclusiones]{Conclusiones y trabajo futuro}{Conclusiones y trabajo futuro}\label{Conclusion}

\noindent
\rule{0.49\textwidth}{0.75pt} $_{\bigcirc}$ \rule{0.49\textwidth}{0.75pt}\\

A pesar de la alta volatilidad del bitcoin se mostró que tratar de predecir series temporales con alta variabilidad y sin supuestos estadísticos tradicionales vale la pena ya que la inteligencia artificial y nuevos métodos matemáticos como el TDA pueden extraer información oculta a simple vista. La combinación de ambas herramientas es una tarea necesaria para enfrentar nuevos retos y en este caso aprender del pasado para predecir el futuro.

\noindent
\rule{0.49\textwidth}{0.75pt} $_{\bigcirc}$ \rule{0.49\textwidth}{0.75pt}\\
\clearpage

\section{Conclusiones}


Las criptomonedas son activos muy volátiles, durante el transcurso de esta tesis el valor del bitcoin ha variado entre \$30,000 y \$70,000 dólares, siendo un activo de inversión de alto riesgo, sin embargo, grandes empresas financieras han empezado a aceptar las criptomonedas como medio de pago, otras han promovido su uso a través del arte digital, la web3.0 o el metaverso, incluso se han desarrollado variantes llamadas stablecoins, que cuentan con el respaldo de la blockchain y basan su valor en una activo consolidado. Las criptomonedas y su tecnología han llegado para quedarse, y los resultados de esta tesis van en la dirección de un futuro donde la adopción ha sido complementaria al sistema fiat o sustituida completamente.

Viendo los resultados obtenidos es razonable suponer que algunas características de la blockchain pueden ayudar a predecir mejor el valor del bitcoin, en este caso, con el método propuesto se obtiene que el hashrate es la métrica que más influye en la varianza y que además está altamente correlacionada con el precio, además se caracteriza por medir la velocidad media a la que los mineros resuelven hashes durante el día. Esto puede interpretarse como que el precio aumenta si la adopción del bitcoin es mayor y más mineros se encuentran dando soporte a la cadena de bloques, por otra parte, también se puede decir que el valor de la moneda en parte se debe a los recursos computacionales utilizados al validar la prueba de trabajo, esto es, resolver hashes. Este resultado coincide con los trabajos de Ji y col. \parencite*{jiBestFeatureSelection2019} y Saad y Mohaisen \parencite*{saadCharacterizingBlockchainbasedCryptocurrencies2018} donde también encontraron con sus respectivos métodos que el hashrate es una característica de valor en la predicción del precio. 

En lo que respecta a la predicción se mostró también que el modelo LSTM, al menos con la configuración dada, no da los mismos resultados que en el estado del arte, concluyendo que, o los nuevos datos hacen reducir su nivel de predicción o hay distintas arquitecturas que brindan mejores resultados, sin embargo, después de hacer una comparación entre 45 modelos, tomando en cuenta las transformaciones matemáticas de los datos y distintos modelos estadísticos univariados y de aprendizaje máquina, se encontró que el modelo que mejor ajusta los datos basados en el RMSE y el índice Theil´s U fue random forest con una transformación logarítmica sobre los datos y con un margen de error de \$38.63 dólares sobre el conjunto de entrenamiento incluyendo métricas financieras y la obtenida del análisis de la blockchain, pudiéndose mostrar que aunque haya mucha variablidad todavia es viable buscar modelos que mejoren la predicción del precio ya sea con otras características de la blockchain o aplicando análisis de sentimiento sobre tweets o noticias.

El análisis topológico de datos fue una herramienta de gran ayuda, ya que permite detectar cambios abruptos en la forma de los datos sin requerir de supuestos estadísticos, en este caso los cambios abruptos en una serie de tiempo financiera como la del precio del bitcoin son los cracs económicos. Estos se caracterizan por un crecimiento constante del precio a lo largo del tiempo y que en algún momento dado empiezan a ocurrir pequeñas caídas rápidas en un periodo corto de tiempo, llevando su valor a mínimos como en el famoso crac del 2018. Cabe mencionar que es una herramienta guía y que en la actividad humana referente a sistemas económicos siempre reina la incertidumbre total, por tanto, si el objetivo es invertir, siempre es recomendable estar atento a las noticias, ya que nunca se sabe cuando pueda surgir una pandemia, una guerra o un desastre natural.

Por último, la metodología de clasificación para la toma de decisiones de inversión mostró ser efectiva utilizando como características la entropía de Shannon y la norma $C^1$ generada a partir del TDA, ya que detecta satisfactoriamente eventos importantes del pasado como los distintos cracs y precios mínimos de la serie de tiempo estudiada, además de clasificar correctamente en un 95.76\% de la veces los datos del conjunto de prueba (imágenes en este caso) con su respectivo cluster de inversión, esperando que mantenga la misma calidad de predicción para eventos futuros. De este modo se espera que inversionistas casuales, de tecnologías financieras, bancarios o de fondos para el retiro pueden tener una guía que ayude a complementar su experiencia y los sistemas actuales de inversión.

En general se espera que la metodología total propuesta ayude a inversionistas con inversiones a mediano y largo plazo con distintas herramientas para encausar sus decisiones a la hora de comprar o vender. Al ser una tecnología en crecimiento hay un campo de desarrollo potencial, por tanto se pretende en trabajos futuros desarrollar los siguientes puntos

\begin{itemize}
	\item Optimizar el ajuste de hiperparámetros de los modelos de machine learning para un predicción óptima.
	\item Aplicar la misma metodología de análisis de métricas a otras criptomonedas populares para entender el comportamiento de las mismas basadas en sus propias blockchains y comparar las métricas de mayor influencia con las del bitcoin.
	\item Automatizar la selección de clusters de inversión tomando en cuenta el signo de la derivada de los datos para comprar, vender o arriesgar.
	\item Agregar variables a través de análisis de sentimiento para ver si pueden mejorar la predicción del precio y la toma de decisiones de inversión.
	\item Agregar análisis financiero como los indicadores bursátiles de soporte y resistencia para determinar mínimos y máximos del precio y así mejorar la toma de decisiones de inversión. 
\end{itemize}
















