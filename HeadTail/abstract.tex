\chapter*{Abstract}

The bitcoin is a cryptocurrency that has acquired great popularity and its price has increased enormously, however it has a great volatility and it becomes necessary to predict its behavior. In the present work, from blockchain features downloaded from February 2, 2012 to April 14, 2022 using the coinmetrics API we analyzed those with the greatest impact on bitcoin price using regression analysis, PCA and correlation dendrogram.

With the characteristics obtained from the blockchain and with the bitcoin price data calculated from the median of three different sources (Yahoo Finance, Investing and Coingecko) with an interval equal to the blockchain metrics we show the results of applying statistical methods such as Naive, SES, Holt, ETS, ARIMA and machine learning methods such as RTS, SVM, RF and LSTM, to predict the bitcoin price with the aim of performing a comparison between methods based on the RMSE error index and Theil's U efficiency index. The price data are transformed by log, BoxCox, diff and diff(log) functions as well as the original scale and a comparative table will be shown with the error and accuracy indexes by transformation to choose the proposed model.

Similarly, the geometry of the data is studied using topological data analysis to deduce that from the $C^1$ norm generated from the persistence diagram, severe price drops can be anticipated, and financial crash alerts can be created.

Finally, with the help of the topological feature and the calculation of the Shannon entropy of the bitcoin price, investment clusters are obtained for image labeling, these latter elaborated from the financial time series, which are classified for investment decision making using CNN. The generated model is used to give a buy, sell or risk recommendation and improve future investment models in the medium and long term.

It is concluded that there are metrics that influence the price variance and also improve the prediction models, on the other hand, the topological analysis of data helps to successfully detect price drops, it also helps to create investment clusters together with Shannon entropy to improve investment decision making.








